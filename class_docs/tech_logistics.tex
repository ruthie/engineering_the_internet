\documentclass{article}
\title{Technology Logistics}
\begin{document}
\maketitle

To do the assignments for this class, you will need access to a computer running Linux or Mac OS.  You are welcome to use either a personal computer or MIT's Athena computers throughout the summer.  You will probably find it most convenient to use either Athena both at home and it MIT, or to use a personal computer both at home and at MIT because it saves you moving files back and forth as well as learning two, slightly different systems.  If you do decide to use both Athena computers and a personal computer, you are responsible for moving files back and forth.

\section*{Athena Computers}
During class, you will have access to MIT's Athena computers.  We have one account on these computers for you to share.  However, outside of class you won't generally have access to these computers.  I can occasionally give you access during dinner, but you shouldn't count on this as homework time.

It is possible to access MIT's Athena computers remotely.  This requires a reasonably fast internet connection (a cable internet connection should be fine).

Each student will get a directory on Athena to put their files in.  Please make every effort to save all of your work in that directory and none anywhere else.  I can't guarantee that files saved elsewhere won't be lost or deleted.

\section*{Personal Computers}
You can also use a personal or family computer.  This computer should be running (ideally) Linux or (workably) Mac OS.  If you have a Windows computer, I recommend installing a virtual machine, which will allow you to have a mini Linux computer run on your regular computer.  I will be available at dinner on the second day to help you install a virtual machine if you don't know how to do so yourself.

If you choose to bring a personal computer to Junction, you are obviously completely responsible for its safety.  Do NOT leave your computer unattended.  Laptop theft at MIT is common, and laptops left in any space that isn't locked are liable to disappear.

\section*{Setting up a Virtual Machine}
A virtual machine can allow you to run linux within a Windows computer.  In comparison to having a dedicated computer, a virtual machine will be less convenient and slower (and it will slow down your Windows computer while it's running).

If you need to set up a virtual machine, I can help you do so on the second day of class.  If you want to try and set it up on your own ahead of time, you can look at the instructions at \texttt{https://help.ubuntu.com/community/VirtualBox} in the section on installing VirtualBox for Windows.  Create a virtual machine running the Ubuntu (linux) operating system version 12.04, 12.10, or 13.04.  If you don't want to try on your own, or if you try and get stuck, don't worry, we'll work it out in person once Junction starts.

\end{document}
