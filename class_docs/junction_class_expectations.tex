\documentclass{article}
\title{Engineering the Internet Class Expectations}
\begin{document}
\maketitle

\section{Goals}
\begin{itemize}
\item Know how the pieces of the internet-system fit together
\item Use basic tools
\item Think about engineering problems
\item Know how to learn more
\end{itemize}

\section{Class Website}
All class materials will be posted on the class website:\\ 
\texttt{http://engineeringtheinternet.nfshost.com/}

\section{Lectures}
There will be lectures most days in class.  It will behoove you to take notes during lecture.  If you have a laptop, don't use it while I lecture.

\section{Homework}

\begin{itemize}
\item Homework will typically be assigned and due on Mondays.
\item Unless noted otherwise, you should email homework to me (ruthie@alum.mit.edu) some time before class on the day it is due.
\item We will go over some homework problems in class.
\item I will read your homework and give you individual feedback within a week.
\item You won't get letter grades for homework (or anything else).
\item Each person needs to turn in their own homework, but you may work with other students on the homework.
\item If you share a significant amount of the effort of doing the homework with another student, please tell me the name of that student and what you worked on with them and how.
\end{itemize}

\section{Website Project}
During the last half of the class you will get the chance to construct and (optionally) publish a simple website of your design.  ESP will pay for a startup-fund for hosting using nearlyfreespeech.net.  The startup fund should be enough to do all the work for the class and keep a simple website up for a year.

\section{Getting help outside of class}
There is no textbook for this class.  If you need help outside of class, you have a number of other resources:
\begin{itemize}
\item Me (ruthie@alum.mit.edu)
\item Lecture notes posted on the website (\texttt{http://engineeringtheinternet.nfshost.com/})
\item Each-other
\item Additional resources and interesting additional reading supplied by me for each unit
\item Google
\end{itemize}

%% \section{Jargon}
%% I spend a large percentage of my time with people who program computers for a living, or expect to in the near future, and sometimes my technical vocabulary gets carried away with itself.  If I use a term you don't understand, please ask what it means.  

%% \begin{itemize}
%% \
%% \end{itemize}

%% To encourage this type of question asking, each time you ask me to define a term, you earn one cookie.  Cookies are not redeamable on the spot, and pastry-cookies may be replaced with computer-cookies if I feel you are being frivolous.

\section{``Easy''}
This class contains students of a variety of backgrounds and skillsets.  Everything we're learning in this class can be called ``easy'' or ``obvious'' or ``trivial'' once you understand it, but not before.  Therefore, if you use any of these words (or synonyms) to describe what we're learning, you are volunteering to help anyone who doesn't understand it yet.  Which is not to say that you should never use those words, just be ready to share your privilege if you do.

\end{document}
